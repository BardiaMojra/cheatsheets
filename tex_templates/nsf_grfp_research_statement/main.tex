% --------------- 12 POINT FONT -------------------------------
\documentclass[12pt]{article}
% --------------- 10 POINT FONT FOR CAPTIONS ------------------
\usepackage[font=footnotesize]{caption}
% --------------- NY TIMES FONT -------------------------------
\usepackage{times}
% --------------- 1 INCH MARGINS ------------------------------
\usepackage[margin=1in]{geometry}
% --------------- LINE SPACING --------------------------------
\usepackage{setspace}
\singlespacing
%\doublespacing
% --------------- SMALL SECTION TITLES ------------------------
\usepackage[tiny,compact]{titlesec}
% --------------- MATH PACKAGES -------------------------------
\usepackage{amsmath,amsthm,amssymb}
\begin{document}




% --------------- TITLE AND NAME ------------------------------
\begin{center}
\textbf{NSF Graduate Research Fellowship}\\
Bardia Mojra
\hfill
Research Statement \\
\large{\bf Robust Systems for Elastic Rod Manipulation } \\
\end{center}
% --------------- CONTENT -------------------------------------

\noindent
Keywords: deformable linear objects, elastic rods, wir harnessing, agile manufacturing, control theory, deep learning\\
\newline

\noindent
\textbf{Evidence of Intellectual Merit can be found in all parts of the application:
Personal Statement, Research Plan, letters, experiences, awards, achievements, and transcripts.}


\noindent
\textbf{Note that your intellectual merit is different from your research’s intellectual merit}\\


\noindent
\textbf{Background:}
%With the rise of the electric vehicle,
%Automation and advanced manufacturing have been focus of intense research as these core technologies have found their way to the production line.

Automotive wiring system has a tree-like shape with thousands wire pieces, components,
terminals and etc \cite{}.


Due to low and fluctuating volumes, automotive manufacturers tend to implement some form of JIT (just-in-time) production to avoid logistical complications.


In recent years, there has been a growing interest in agile manufacturing research and technologies with robotic manipulation be among the highly researched topics. Although robotic manipulators have found a prominent place on the production line, the automotive wire harnessing task remains up to 90\% dependant on manual labor \cite{nguyen2021manufacturing}.



\noindent
\textbf{Bridge Background to Problem Statement}
Degrees of automation deployed varies among the three stages with the first being most heavily automated and the last being the least automated. The human-machine production rate disparity is clear here, 70\% of wire harness manufacturing time is spent on the final step and in some areas it is up to 90\% manual labor \cite{nguyen2021manufacturing}.\\


Current elastic rod manipulation approaches could be categorized to two main groups, classical and data-driven. Classical methods such as...\\



\noindent
\textbf{Intellectual Merit (rm sec):} The Intellectual Merit criterion encompasses the potential to advance knowledge.\\
What is the potential for the proposed activity to:
Advance knowledge and understanding within its own field or across different fields (Intellectual Merit)?\\
Demonstrated intellectual ability (grades, curricula, awards, publications, presentations, etc.)
Other evidence of your potential, such as ability to:\\
Plan and conduct research\\
Work as a member of a team as well as independently\\
Interpret and communicate research\\
Take initiative, solve problems, persist.\\

\noindent
\textbf{Problem Statement:}

\noindent
\textbf{Question 1 - Current Limitations:}

\noindent
\textbf{Question 2 - Knowledge Gaps:}

\noindent
\textbf{Question 3 - }

Moreover, it is essential to develop easily reconfigurable wire harnessing manufacturing processes and systems.


\noindent
Despite the lack of long research history in this area, recent development have shown promising results. [cite cable harnessing papers]\\

Relative developments that makes my proposal ideal next step..\\

Multiple model based and model-free systems have been develop for robotic are
manipulation.
Current limitations?
How can our contribution advance SOTA? 
Assumption is that the research is not done yet! We don’t know the solution and that's okay.

mention the gap\\
how are we training\\
Cost functions and rewards\\









\noindent
\textbf{Research Plan:} To what extent do the proposed activities suggest and explore creative, original, or potentially transformative concepts?\\
Is the plan for carrying out the proposed activities well-reasoned, well-organized, and based on a sound rationale? Does the plan incorporate a mechanism to assess success?\\
\noindent
\textbf{Aim 1} You can type your Aim 1 here if you have one. If you don't then you can delete the subsection\\
\noindent
\textbf{Aim 2} You can type your Aim 1 here if you have one. If you don't then you can delete the subsection\\



%https://www.nist.gov/programs-projects/model-based-systems-definition-and-analysis-integration-smart-manufacturing

\noindet
WHAT IS THE RESEARCH PLAN?
\noindet
Systems models capture customer requirements, high-level system specifications, verification tests, and the relationships between them. They gather overall systems information in one format, improving consistency and reducing redundancy. The project will use overall system models to coordinate discipline-specific engineering analysis by identifying and eliminating inconsistencies between systems and analysis models, and between analysis models themselves. The high-level plan is:\\
\noindet
Identify discipline-specific analysis methods and tools useful in manufacturing development operations. Focusing the project on these analysis methods and tools will provide the greatest impact for integration. The selected analysis methods currently are discrete event logistics/production simulation and optimization, finite element analysis, and behavior verification.\\
\noindet
Identify redundancies and inconsistencies between systems models and the selected analysis methods and tools.\\
\noindet
Develop methods and protocols that prevent redundancies and inconsistencies between systems and analysis information, either by maintaining links between systems and analysis information, or by consolidating analysis information in overall systems models.\\
\noindet
Develop methods and protocols that prevent redundancies and inconsistencies between analysis information on multiple tools, by:\\
\noindet
identifying and abstracting commonalities among specific kinds of discipline-specific analysis information that are useful in smart manufacturing operations\\
\noindet
maintaining links between information on multiple analysis tools, or by consolidating it in extensions of overall systems models, based on the abstractions above.\\


\noindent
\textbf{WHAT IS THE RESEARCH?}
Control algorithm?
Apply this to that to produce

INVESTIGATE DATA DRIVEN MODEL FOR CABLE MANIPULATION.

WE ARE ANSWERING THESE 3 QUESTIONS

we will try this approaches.




\noindent
\textbf{Broader Impacts (rm sec):} The Broader Impacts criterion encompasses the potential to benefit society and contribute to the achievement of specific, desired societal outcomes.\\
To promote the progress of science\\
To advance the national health, prosperity, and welfare\\
To secure the national defense\\
What benefit society or advance desired societal outcomes (Broader Impacts)?\\
To what extent do the proposed activities suggest and explore creative, original, or potentially transformative concepts?\\
Societal benefits may include, but are not limited to:\\
Increasing participation of underrepresented groups, women, persons with disabilities, veterans\\
Outreach: Mentoring; improving STEM education in schools\\
Increasing public scientific literacy; increased public engagement with STEM\\
Community outreach: science clubs, radio, TV, newspapers, blogs\\
Increasing collaboration between academia, industry, others\\
\noindent
\textbf{Evidence of Broader Impacts can be in all parts of the application: Personal Statement, Research Plan, letters, experiences, awards, achievements}\\

\noindent
Add injuring and repetitive

\noindent
Results of this research will benef


\noindent
\textbf{Qualifications:} How well qualified is the individual, team, or organization to conduct the proposed activities? \\
How did you learn about this field? E.g. through classes, readings, seminars, work or other experiences, or conversations with people already in the field?\\
How have you capitalized on opportunities available to you?\\
What reasons can you give for reviewers to be interested in your application?\\
What impact have you had on your academic, local and the broader community?\\



\noindent
\textbf{Available Resources:} Are there adequate resources available to the PI (either at the home organization or through collaborations) to carry out the proposed activities?\\




\noindent
\textbf{Notes from citations: -- DO NOT COPY}

\noindent
\textbf{Notes from other funded proposals: -- DO NOT COPY}
- Collaborative Research: Adaptive, Rapid, and Multifunctional Soft Robots (ARM SoRo) with Reconfigurable Shapes and Motions Enabled by Tunable Elastic Instabilities: "Additionally, this award will offer a unique opportunity to integrate insights from robotics, mechanics, design, and fabrication into intellectually intriguing and visually appealing broadening participation activities to inspire, engage, and educate students and the public alike, with the science and technology of reconfigurable robots. Examples of activities include senior design projects, summer program for high school students, and science and engineering festival."







% --------------- WORKS CITED (10pt FONT) ---------------------

\footnotesize
\pagenumbering{arabic}
\renewcommand{\thepage} {E--\arabic{page}}

\bibliography{ref}
\bibliographystyle{ieee}

\end{document}

% -------------------------------------------------------------

% -------------------------------------------------------------
